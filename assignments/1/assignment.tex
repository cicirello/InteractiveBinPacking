% Copyright 2021 Vincent A. Cicirello
%
% License: CC BY-NC-SA (Creative Commons Attribution-NonCommercial-ShareAlike)
% Full license text link: https://creativecommons.org/licenses/by-nc-sa/4.0/
% Short summary of license: Allows remixing and adapting non-commercially, 
% requires attribution, and adaptations required to be licensed under 
% identical terms.
%
% Instructors: If you are an instructor of a course using this assignment
% keep an eye out for comments addressed to you throughout this file.

\documentclass[11pt,letterpaper]{article}

\usepackage{times}
\usepackage{url}
\usepackage{graphicx}
\usepackage{hyperref}
\usepackage{enumitem}

\setlength{\textwidth}{6.5in}
\setlength{\oddsidemargin}{0in}
\setlength{\evensidemargin}{0in}
\setlength{\topmargin}{0in}
\setlength{\headheight}{0in}
\setlength{\headsep}{0in}
\setlength{\textheight}{9in}

\begin{document}

\title{Combinatorial Optimization, the Bin Packing Problem, 
and Constructive Heuristics}

% Instructors: You might use the author field for the name
% of the course or perhaps the assignment number. I've
% left the author blank for now.
\author{}

% Instructors: You might insert the date of the assignment
% here. I've left the date blank for now.
\date{}

\maketitle

% Instructors: You might want to insert a section with your
% own special instructions, such as how to submit the assignment,
% deadline, etc.

\section{Objectives, Prerequisites, and Time Requirements}\label{sec:objectives}
This assignment will begin with a self-guided tutorial 
(Section~\ref{sec:tutorial}) that will introduce you to combinatorial 
optimization. This includes learning about what combinatorial optimization
problems are in general, as well as learning about a specific 
combinatorial optimization problem, the {\em Bin Packing Problem},
used by the tutorial to illustrate concepts such as lower bounds
as well as how to use heuristics to quickly compute satisficing
(though not necessarily optimal) solutions. The content of the tutorial
is entirely within a Java application (instructions for downloading
and running are found in Section~\ref{sec:download}).

After you finish working through the tutorial (Section~\ref{sec:tutorial}),
you will work through four exercises 
(Sections~\ref{sec:ff},~\ref{sec:ffd},~\ref{sec:bf}, and~\ref{sec:bfd}) 
to test your knowledge of the constructive heuristics that you learned 
about for the bin packing problem.

\paragraph*{Objectives:} The objectives of this assignment include:
\begin{itemize}[leftmargin=*, parsep=0pt, itemsep=2pt, topsep=2pt]
\item gaining a general understanding of combinatorial optimization problems;
\item gaining an understanding of the bin packing problem, its 
applications, and how it is an example of a combinatorial 
optimization problem;
\item learning about lower bounds;
\item learning about constructive heuristics; and
\item learning about the most common constructive heuristics 
for the bin packing problem, including first-fit, best-fit, 
first-fit decreasing, and best-fit decreasing.
\end{itemize}

\paragraph*{Prerequisite Knowledge:} The tutorial does not assume
any prior knowledge of combinatorial optimization. However, it does 
assume that you have an understanding of introductory discrete mathematics,
such as set theory and functions.

\paragraph*{Technical Prerequisites:} This assignment does not require
programming. However, you will need to be able to run a Java application.
The Java application in question requires Java 11 or later. You do not need
a Java JDK as we will not be compiling the Java program. Only a Java Runtime
Environment (JRE) is needed.

\paragraph*{Time Requirements:} The time required to complete this assignment
including the time to work through the self-guided tutorial may vary by
student depending upon prior background, course level, etc. We estimate that
the time required is between 66 minutes and 135 minutes.

\paragraph*{Note to Instructors:} This assignment can potentially be used
in courses on: (a) discrete mathematics toward the end to introduce 
combinatorial optimization, (b) algorithms to provide an example of an
NP-Hard problem and how heuristics can sometimes be used to efficiently
compute satisficing solutions despite the problem's complexity, or (c)
artificial intelligence as an example of heuristic problem solving or
just prior to coverage of local search algorithms.

\paragraph*{Licensing:} The {\em Interactive Bin Packing} application,
which you will download, is open source and licensed under the GPLv3 
(\url{https://www.gnu.org/licenses/gpl-3.0.en.html}). This assignment,
which utilizes the application, is licensed under the CC BY-NC-SA
(\url{https://creativecommons.org/licenses/by-nc-sa/4.0/}). 


\section{Download Interactive Bin Packing (1--5 minutes)}\label{sec:download}
In this assignment, we'll be using a Java application 
called {\em Interactive Bin Packing}. It is open source and
can be found at this 
link: \url{https://github.com/cicirello/InteractiveBinPacking}.
For this assignment, we won't need the source code of the application.
Instead, we will use the jar file of the most recent release.
Begin by downloading the {\em Interactive Bin Packing}
application. Find instructions for downloading
here: \url{https://github.com/cicirello/InteractiveBinPacking#installing-and-running-the-application}. 

Note that in order to run the application you will need
a Java Runtime Environment (JRE), version 11 or later. Since we will
not be building from the source, you do not need a JDK---a JRE is sufficient.
If you are not sure if you have Java installed, just try to run the 
application. 

If you encounter any issues with the application, you are encouraged to
report bugs and other issues via the issue tracker: 
\url{https://github.com/cicirello/InteractiveBinPacking/issues}.


\section{Complete the Tutorial (45--90 minutes)}\label{sec:tutorial}
Before proceeding to specific bin packing problems, we'll start
by working through a tutorial. At any point, you can stop and return 
later. Do the following:
\begin{enumerate}[leftmargin=*, parsep=0pt, itemsep=2pt, topsep=2pt]
\item Now that it has been downloaded, run the {\em Interactive Bin Packing}
application. It is an executable jar file, so you can either just double click
it, or you can run it from the command line.  See this link if you need
help running: \url{https://github.com/cicirello/InteractiveBinPacking#running}.
\item Click on {\em Info} menu and then the {\em Tutorial} option in that menu.
\item I recommend moving the Tutorial window that appears to the 
side of the application so that you can see both.
\item Read through the content of the Tutorial window in its 
entirety, while working along with it in the application. The tutorial 
will explain the Bin Packing problem, examples of a few applications 
of it, and then it will explain the most common constructive heuristics 
for the problem. I especially recommend using the application to 
follow along when you get to the heuristics, including switching the mode
as indicated so that the application can help ensure that you are 
correctly understanding how the heuristics work.
\end{enumerate}


\newpage

\section{First-Fit Heuristic (5--10 minutes)}\label{sec:ff}

% TIPS TO INSTRUCTORS:
% (1) Feel free to change the instance number in the first step below.
%     There isn't anything special about that one.
% (2) If you are concerned with the possibility that students
%     might share with friends who take the course in a future
%     semester, then perhaps use a difference problem instance
%     each semester.
% (3) If you are concerned with students sharing with each 
%     other in the same semester, then perhaps assign each student
%     a different instance number (perhaps have them use their
%     student id number as the instance number).

You will now compute the solution determined by the First-Fit
heuristic for a specific bin packing instance. Do the following:
\begin{enumerate}[leftmargin=*, parsep=0pt, itemsep=2pt, topsep=2pt]
\item From the {\em Problem} menu, choose {\em Select Instance Number}
to choose instance number \textbf{5}.
\item Compute the solution that would be found using the First-Fit 
heuristic. Use the application for guidance by switching the mode.
It won't let you make mistakes.
\item Answer the following questions.
\end{enumerate}

\paragraph*{What solution did it find?} Specifically, for each bin in the
solution, list the items in that bin (using the single-letter names) in the
order that they were added to the bin. You may answer this by providing a
screen shot if you'd like.

\vspace*{2in}

\paragraph*{What is the lower bound for this instance?} Recall that the
{\em Operations} menu has a command that will compute this for you.

\vspace*{0.25in}

\paragraph*{Do we know if this solution is optimal?} Do we have enough 
information to determine with certainty whether the solution found by
this heuristic is the optimal solution (i.e., without doing any additional
computation)? If yes, explain why. If no, explain why we don't know for sure.

\vspace*{1in}


\newpage

\section{First-Fit Decreasing Heuristic (5--10 minutes)}\label{sec:ffd}

% TIPS TO INSTRUCTORS:
% (1) Feel free to change the instance number in the first step below.
%     There isn't anything special about that one.
% (2) If you are concerned with the possibility that students
%     might share with friends who take the course in a future
%     semester, then perhaps use a difference problem instance
%     each semester.
% (3) If you are concerned with students sharing with each 
%     other in the same semester, then perhaps assign each student
%     a different instance number (perhaps have them use their
%     student id number as the instance number).

You will now compute the solution determined by the First-Fit Decreasing
heuristic for a specific bin packing instance. Do the following:
\begin{enumerate}[leftmargin=*, parsep=0pt, itemsep=2pt, topsep=2pt]
\item From the {\em Problem} menu, choose {\em Select Instance Number}
to choose instance number \textbf{5}.
\item Compute the solution that would be found using the First-Fit Decreasing
heuristic. Use the application for guidance by switching the mode.
It won't let you make mistakes.
\item Answer the following questions.
\end{enumerate}

\paragraph*{What solution did it find?} Specifically, for each bin in the
solution, list the items in that bin (using the single-letter names) in the
order that they were added to the bin. You may answer this by providing a
screen shot if you'd like.

\vspace*{2in}

\paragraph*{What is the lower bound for this instance?} Recall that the
{\em Operations} menu has a command that will compute this for you.

\vspace*{0.25in}

\paragraph*{Do we know if this solution is optimal?} Do we have enough 
information to determine with certainty whether the solution found by
this heuristic is the optimal solution (i.e., without doing any additional
computation)? If yes, explain why. If no, explain why we don't know for sure.

\vspace*{1in}


\newpage

\section{Best-Fit Heuristic (5--10 minutes)}\label{sec:bf}

% TIPS TO INSTRUCTORS:
% (1) Feel free to change the instance number in the first step below.
%     There isn't anything special about that one.
% (2) If you are concerned with the possibility that students
%     might share with friends who take the course in a future
%     semester, then perhaps use a difference problem instance
%     each semester.
% (3) If you are concerned with students sharing with each 
%     other in the same semester, then perhaps assign each student
%     a different instance number (perhaps have them use their
%     student id number as the instance number).

You will now compute the solution determined by the Best-Fit
heuristic for a specific bin packing instance. Do the following:
\begin{enumerate}[leftmargin=*, parsep=0pt, itemsep=2pt, topsep=2pt]
\item From the {\em Problem} menu, choose {\em Select Instance Number}
to choose instance number \textbf{5}.
\item Compute the solution that would be found using the Best-Fit 
heuristic. Use the application for guidance by switching the mode.
It won't let you make mistakes.
\item Answer the following questions.
\end{enumerate}

\paragraph*{What solution did it find?} Specifically, for each bin in the
solution, list the items in that bin (using the single-letter names) in the
order that they were added to the bin. You may answer this by providing a
screen shot if you'd like.

\vspace*{2in}

\paragraph*{What is the lower bound for this instance?} Recall that the
{\em Operations} menu has a command that will compute this for you.

\vspace*{0.25in}

\paragraph*{Do we know if this solution is optimal?} Do we have enough 
information to determine with certainty whether the solution found by
this heuristic is the optimal solution (i.e., without doing any additional
computation)? If yes, explain why. If no, explain why we don't know for sure.

\vspace*{1in}

\newpage

\section{Best-Fit Decreasing Heuristic (5--10 minutes)}\label{sec:bfd}

% TIPS TO INSTRUCTORS:
% (1) Feel free to change the instance number in the first step below.
%     There isn't anything special about that one.
% (2) If you are concerned with the possibility that students
%     might share with friends who take the course in a future
%     semester, then perhaps use a difference problem instance
%     each semester.
% (3) If you are concerned with students sharing with each 
%     other in the same semester, then perhaps assign each student
%     a different instance number (perhaps have them use their
%     student id number as the instance number).

You will now compute the solution determined by the Best-Fit Decreasing
heuristic for a specific bin packing instance. Do the following:
\begin{enumerate}[leftmargin=*, parsep=0pt, itemsep=2pt, topsep=2pt]
\item From the {\em Problem} menu, choose {\em Select Instance Number}
to choose instance number \textbf{5}.
\item Compute the solution that would be found using the Best-Fit Decreasing
heuristic. Use the application for guidance by switching the mode.
It won't let you make mistakes.
\item Answer the following questions.
\end{enumerate}

\paragraph*{What solution did it find?} Specifically, for each bin in the
solution, list the items in that bin (using the single-letter names) in the
order that they were added to the bin. You may answer this by providing a
screen shot if you'd like.

\vspace*{2in}

\paragraph*{What is the lower bound for this instance?} Recall that the
{\em Operations} menu has a command that will compute this for you.

\vspace*{0.25in}

\paragraph*{Do we know if this solution is optimal?} Do we have enough 
information to determine with certainty whether the solution found by
this heuristic is the optimal solution (i.e., without doing any additional
computation)? If yes, explain why. If no, explain why we don't know for sure.

\vspace*{1in}


\end{document}