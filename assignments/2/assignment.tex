% Copyright 2021 Vincent A. Cicirello
%
% License: CC BY-NC-SA (Creative Commons Attribution-NonCommercial-ShareAlike)
% Full license text link: https://creativecommons.org/licenses/by-nc-sa/4.0/
% Short summary of license: Allows remixing and adapting non-commercially, 
% requires attribution, and adaptations required to be licensed under 
% identical terms.
%
% Original version of this assignment can be found in the assignments directory
% in this GitHub repository: https://github.com/cicirello/InteractiveBinPacking
%
% Instructors: If you are an instructor of a course using this assignment
% keep an eye out for comments addressed to you throughout this file.

\documentclass[11pt,letterpaper]{article}

\usepackage{times}
\usepackage{url}
\usepackage{graphicx}
\usepackage{hyperref}
\usepackage{enumitem}

\setlength{\textwidth}{6.5in}
\setlength{\oddsidemargin}{0in}
\setlength{\evensidemargin}{0in}
\setlength{\topmargin}{0in}
\setlength{\headheight}{0in}
\setlength{\headsep}{0in}
\setlength{\textheight}{9in}

\begin{document}

\title{Combinatorial Optimization, the Bin Packing Problem, 
and Constructive Heuristics}

% Instructors: You might use the author field for the name
% of the course or perhaps the assignment number. I've
% left the author blank for now.
\author{}

% Instructors: You might insert the date of the assignment
% here. I've left the date blank for now.
\date{}

\maketitle

\section{Before We Begin}\label{sec:before}

% Instructors: You might need to customize this section with your
% own special instructions, such as how to submit the assignment,
% deadline, etc. Especially note the paragraph below that indicates
% using student id for the instance number later in the assignment.

\paragraph*{Problem Instance Numbers:}
Later in this assignment, you will work through a few exercises,
and each student's exercises will be different. Specifically, 
each student will use a different instance of the bin packing problem.
When you get to the exercises in Sections~\ref{sec:ff},~\ref{sec:ffd},~\ref{sec:bf}, 
and~\ref{sec:bfd}, you will be asked to enter your problem instance number.
This will be uniquely determined for each student by your student id number.
It must be an integer, so start by dropping any letters or other non-numeric
characters from your student id number. If there are any 0s at the start, then
drop those as well. What remains is your problem instance number.
For example, if your student id number is ABC00012345678900, then your
problem instance number is 12345678900. In the application that you are using
the problem instance numbers are values of type long (in Java).

\paragraph*{Session Logs:}
In the application that we will use, under the ``Session'' menu, you will
find options to save a session log. This will generate a file with an 
extension of ``*.ibp''. You can otherwise name the file anything you'd like.
If you work on the assignment over multiple sessions, just save each of the 
sessions. Make sure you don't select the same file as your previous session,
or else it will be overwritten. Don't attempt to edit the session log files
manually. Doing so will very likely generate alerts when your instructor
views the session logs, possibly invalidating your work.

The session log files are what you will be submitting for this assignment,
in addition to answers to any questions found later in this assignment. If you
spread your work over multiple sessions, then just submit the log files for all
of your sessions.

\paragraph*{What Your Instructor Will be Looking for in the Session Logs:}
The instructor of your course, in addition to the answers to questions later in this
document, will expect to find a record of the following in the session logs that
you submit: (1) that you completed the ``Default'' problem instance in all 4
heuristic modes, when you worked through the complete tutorial in 
Section~\ref{sec:tutorial}; and (2) that you completed your unique problem instance
when you worked through the parts of the assignment from 
Sections~\ref{sec:ff},~\ref{sec:ffd},~\ref{sec:bf}, and~\ref{sec:bfd}.

\section{Objectives, Prerequisites, and Time Requirements}\label{sec:objectives}
This assignment will begin with a self-guided tutorial 
(Section~\ref{sec:tutorial}) that will introduce you to combinatorial 
optimization. This includes learning about what combinatorial optimization
problems are in general, as well as learning about a specific 
combinatorial optimization problem, the {\em Bin Packing Problem},
used by the tutorial to illustrate concepts such as lower bounds
as well as how to use heuristics to quickly compute satisficing
(though not necessarily optimal) solutions. The content of the tutorial
is entirely within a Java application (instructions for downloading
and running are found in Section~\ref{sec:download}).

After you finish working through the tutorial (Section~\ref{sec:tutorial}),
you will work through four exercises 
(Sections~\ref{sec:ff},~\ref{sec:ffd},~\ref{sec:bf}, and~\ref{sec:bfd}) 
to test your knowledge of the constructive heuristics that you learned 
about for the bin packing problem.

\paragraph*{Objectives:} The objectives of this assignment include:
\begin{itemize}[leftmargin=*, parsep=0pt, itemsep=2pt, topsep=2pt]
\item gaining a general understanding of combinatorial optimization problems;
\item gaining an understanding of the bin packing problem, its 
applications, and how it is an example of a combinatorial 
optimization problem;
\item learning about lower bounds;
\item learning about constructive heuristics; and
\item learning about the most common constructive heuristics 
for the bin packing problem, including first-fit, best-fit, 
first-fit decreasing, and best-fit decreasing.
\end{itemize}

\paragraph*{Prerequisite Knowledge:} The tutorial does not assume
any prior knowledge of combinatorial optimization. However, it does 
assume that you have an understanding of introductory discrete mathematics,
such as set theory and functions.

\paragraph*{Technical Prerequisites:} This assignment does not require
programming. However, you will need to be able to run a Java application.
The Java application in question requires Java 11 or later. You do not need
a Java JDK as we will not be compiling the Java program. Only a Java Runtime
Environment (JRE) is needed.

\paragraph*{Time Requirements:} The time required to complete this assignment
including the time to work through the self-guided tutorial may vary by
student depending upon prior background, course level, etc. We estimate that
the time required is between 66 minutes and 135 minutes.

\paragraph*{Note to Instructors:} This assignment can potentially be used
in courses on: (a) discrete mathematics toward the end to introduce 
combinatorial optimization, (b) algorithms to provide an example of an
NP-Hard problem and how heuristics can sometimes be used to efficiently
compute satisficing solutions despite the problem's complexity, or (c)
artificial intelligence as an example of heuristic problem solving or
just prior to coverage of local search algorithms.

\paragraph*{Licensing:} The {\em Interactive Bin Packing} application,
which you will download, is open source and licensed under the GPLv3 
(\url{https://www.gnu.org/licenses/gpl-3.0.en.html}). This assignment,
which utilizes the application, is licensed under the CC BY-NC-SA
(\url{https://creativecommons.org/licenses/by-nc-sa/4.0/}). The original
version of this assignment is available 
at \url{https://github.com/cicirello/InteractiveBinPacking}. Students 
should rely on the copy of the assignment provided them by their instructor
as they may have adapted it to your course. 

\section{Download Interactive Bin Packing (1--5 minutes)}\label{sec:download}
In this assignment, we'll be using a Java application 
called {\em Interactive Bin Packing}. It is open source and
can be found at this 
link: \url{https://github.com/cicirello/InteractiveBinPacking}.
For this assignment, we won't need the source code of the application.
Instead, we will use the jar file of the most recent release.
Begin by downloading the {\em Interactive Bin Packing}
application. Find instructions for downloading
here: \url{https://github.com/cicirello/InteractiveBinPacking#installing-and-running-the-application}. 

Note that in order to run the application you will need
a Java Runtime Environment (JRE), version 11 or later. Since we will
not be building from the source, you do not need a JDK---a JRE is sufficient.
If you are not sure if you have Java installed, just try to run the 
application. 

If you encounter any issues with the application, you are encouraged to
report bugs and other issues via the issue tracker: 
\url{https://github.com/cicirello/InteractiveBinPacking/issues}.


\section{Complete the Tutorial (45--90 minutes)}\label{sec:tutorial}
Before proceeding to specific bin packing problems, we'll start
by working through a tutorial. At any point, you can stop and return 
later. Do the following:
\begin{enumerate}[leftmargin=*, parsep=0pt, itemsep=2pt, topsep=2pt]
\item Now that it has been downloaded, run the {\em Interactive Bin Packing}
application. It is an executable jar file, so you can either just double click
it, or you can run it from the command line.  See this link if you need
help running: \url{https://github.com/cicirello/InteractiveBinPacking#running}.
\item Click on {\em Info} menu and then the {\em Tutorial} option in that menu.
\item I recommend moving the Tutorial window that appears to the 
side of the application so that you can see both.
\item Read through the content of the Tutorial window in its 
entirety, while working along with it in the application. The tutorial 
will explain the Bin Packing problem, examples of a few applications 
of it, and then it will explain the most common constructive heuristics 
for the problem. I especially recommend using the application to 
follow along when you get to the heuristics, including switching the mode
as indicated so that the application can help ensure that you are 
correctly understanding how the heuristics work.
\end{enumerate}

Before moving on to the remainder of the assignment, this might
be a good place to save your session log so that you have documentation
that you completed the tutorial above.  You can always save it again
if you continue in this same session, or save an additional one if
you take a break.


\newpage

\section{First-Fit Heuristic (5--10 minutes)}\label{sec:ff}

% TIPS TO INSTRUCTORS:
% (1) Since each student has their own problem instance number,
%     you don't need to worry that anyone will share solutions
%     since doing so would require a student to actually do 
%     all of the work an extra time to solve their friend's instance
%     in addition to their own.
% (2) The application will assist you in verifying that each student
%     completed the assignment. Open the application, and use the
%     Load Session Log command under the session menu to view the
%     student's submitted session logs. That command will validate its
%     contents, assuring that the student didn't attempt to alter it,
%     and will summarize which problem instances they completed in which
%     heuristic modes. You can check there that they actually completed
%     the instance number that corresponds to their student id.

You will now compute the solution determined by the First-Fit
heuristic for a specific bin packing instance. Do the following:
\begin{enumerate}[leftmargin=*, parsep=0pt, itemsep=2pt, topsep=2pt, partopsep=0pt]
\item From the {\em Problem} menu, choose {\em Select Instance Number}
to choose your unique problem instance number determined earlier in 
Section~\ref{sec:before}.
\item Compute the solution that would be found using the First-Fit 
heuristic. Use the application for guidance by switching the mode.
It won't let you make mistakes.
\item Answer the following questions.
\end{enumerate}

\vspace*{-0.25in}

\paragraph*{How many bins did this heuristic use?}

\vspace*{0.25in}

\paragraph*{What is the lower bound for this instance?} Recall that the
{\em Operations} menu has a command that will compute this for you.

\vspace*{0.25in}

\paragraph*{Do we know if this solution is optimal?} Do we have enough 
information to determine with certainty whether the solution found by
this heuristic is the optimal solution (i.e., without doing any additional
computation)? If yes, explain why. If no, explain why we don't know for sure.

\vspace*{0.5in}



\section{First-Fit Decreasing Heuristic (5--10 minutes)}\label{sec:ffd}

% TIPS TO INSTRUCTORS:
% (1) Since each student has their own problem instance number,
%     you don't need to worry that anyone will share solutions
%     since doing so would require a student to actually do 
%     all of the work an extra time to solve their friend's instance
%     in addition to their own.
% (2) The application will assist you in verifying that each student
%     completed the assignment. Open the application, and use the
%     Load Session Log command under the session menu to view the
%     student's submitted session logs. That command will validate its
%     contents, assuring that the student didn't attempt to alter it,
%     and will summarize which problem instances they completed in which
%     heuristic modes. You can check there that they actually completed
%     the instance number that corresponds to their student id.

You will now compute the solution determined by the First-Fit Decreasing
heuristic for a specific bin packing instance. Do the following:
\begin{enumerate}[leftmargin=*, parsep=0pt, itemsep=2pt, topsep=2pt, partopsep=0pt]
\item From the {\em Problem} menu, choose {\em Select Instance Number}
to choose your unique problem instance number determined earlier in 
Section~\ref{sec:before}.
\item Compute the solution that would be found using the First-Fit Decreasing
heuristic. Use the application for guidance by switching the mode.
It won't let you make mistakes.
\item Answer the following questions.
\end{enumerate}

\vspace*{-0.25in}

\paragraph*{How many bins did this heuristic use?}

\vspace*{0.25in}

\paragraph*{What is the lower bound for this instance?} Recall that the
{\em Operations} menu has a command that will compute this for you.

\vspace*{0.25in}

\paragraph*{Do we know if this solution is optimal?} Do we have enough 
information to determine with certainty whether the solution found by
this heuristic is the optimal solution (i.e., without doing any additional
computation)? If yes, explain why. If no, explain why we don't know for sure.

\vspace*{0.5in}


\newpage

\section{Best-Fit Heuristic (5--10 minutes)}\label{sec:bf}

% TIPS TO INSTRUCTORS:
% (1) Since each student has their own problem instance number,
%     you don't need to worry that anyone will share solutions
%     since doing so would require a student to actually do 
%     all of the work an extra time to solve their friend's instance
%     in addition to their own.
% (2) The application will assist you in verifying that each student
%     completed the assignment. Open the application, and use the
%     Load Session Log command under the session menu to view the
%     student's submitted session logs. That command will validate its
%     contents, assuring that the student didn't attempt to alter it,
%     and will summarize which problem instances they completed in which
%     heuristic modes. You can check there that they actually completed
%     the instance number that corresponds to their student id.

You will now compute the solution determined by the Best-Fit
heuristic for a specific bin packing instance. Do the following:
\begin{enumerate}[leftmargin=*, parsep=0pt, itemsep=2pt, topsep=2pt, partopsep=0pt]
\item From the {\em Problem} menu, choose {\em Select Instance Number}
to choose your unique problem instance number determined earlier in 
Section~\ref{sec:before}.
\item Compute the solution that would be found using the Best-Fit 
heuristic. Use the application for guidance by switching the mode.
It won't let you make mistakes.
\item Answer the following questions.
\end{enumerate}

\vspace*{-0.25in}

\paragraph*{How many bins did this heuristic use?}

\vspace*{0.25in}

\paragraph*{What is the lower bound for this instance?} Recall that the
{\em Operations} menu has a command that will compute this for you.

\vspace*{0.25in}

\paragraph*{Do we know if this solution is optimal?} Do we have enough 
information to determine with certainty whether the solution found by
this heuristic is the optimal solution (i.e., without doing any additional
computation)? If yes, explain why. If no, explain why we don't know for sure.

\vspace*{0.5in}



\section{Best-Fit Decreasing Heuristic (5--10 minutes)}\label{sec:bfd}

% TIPS TO INSTRUCTORS:
% (1) Since each student has their own problem instance number,
%     you don't need to worry that anyone will share solutions
%     since doing so would require a student to actually do 
%     all of the work an extra time to solve their friend's instance
%     in addition to their own.
% (2) The application will assist you in verifying that each student
%     completed the assignment. Open the application, and use the
%     Load Session Log command under the session menu to view the
%     student's submitted session logs. That command will validate its
%     contents, assuring that the student didn't attempt to alter it,
%     and will summarize which problem instances they completed in which
%     heuristic modes. You can check there that they actually completed
%     the instance number that corresponds to their student id.

You will now compute the solution determined by the Best-Fit Decreasing
heuristic for a specific bin packing instance. Do the following:
\begin{enumerate}[leftmargin=*, parsep=0pt, itemsep=2pt, topsep=2pt, partopsep=0pt]
\item From the {\em Problem} menu, choose {\em Select Instance Number}
to choose your unique problem instance number determined earlier in 
Section~\ref{sec:before}.
\item Compute the solution that would be found using the Best-Fit Decreasing
heuristic. Use the application for guidance by switching the mode.
It won't let you make mistakes.
\item Answer the following questions.
\end{enumerate}

\vspace*{-0.25in}

\paragraph*{How many bins did this heuristic use?}

\vspace*{0.25in}

\paragraph*{What is the lower bound for this instance?} Recall that the
{\em Operations} menu has a command that will compute this for you.

\vspace*{0.25in}

\paragraph*{Do we know if this solution is optimal?} Do we have enough 
information to determine with certainty whether the solution found by
this heuristic is the optimal solution (i.e., without doing any additional
computation)? If yes, explain why. If no, explain why we don't know for sure.

\vspace*{0.5in}

\textbf{\large IMPORTANT: Make sure you save your session log to submit with your assignment.}
\end{document}